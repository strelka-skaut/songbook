\mysong{Pažitka}{Xavier Baumaxa 2005}{}
\begin{song}{}
\begin{verse}
   Krajina \mchord{Asus2}svádí k podzimním \mchord*{C#m}výl etům, \\
   tripům do \mchord{F#m}mládí a častým \mchord*{Dsus2}úletů m. \\
   Barevný listí je rázem pestřejší, \\
   hlavu ti čistí, no a ty jsi bystřejší.
\end{verse}
\begin{refren}
   |: \mchord{Asus2}Pašuješ zážitky, \mchord{E}pašuješ všechno co se \mchord*{F#m}dá,  \\
   sáčky suchý pažitky\mchord{Dsus2},~tomu se říká dobrá nál\mchord{Asus2}ada. :|
\end{refren}
\begin{verse}
   Hladina \ochord{Asus2}tůní~pomalu \ochord*{C#m}vyc hladá, \\
   je konec \ochord{F#m}vůním a léto \ochord*{Dsus2}uvadá . \\
   Na stehna fenek, dopadl dlouhý stín, \\
   skončil čas trenek, ale já zas něco vymyslím.
\end{verse}
\begin{refren}
   |: \ochord{Asus2}Pašuješ zážitky, \ochord{E}pašuješ všechno co se \ochord*{F#m}dá,  \\
   sáčky suchý pažitky,\ochord{Dsus2}~tomu se říká dobrá nála\ochord{Asus2}da. :|
\end{refren}
\begin{verse}
	Kochám se \ochord{Asus2}\censor{chcaje}, překvapen \ochord*{C#m}\censor{výd rží}, \\
	\censor{té co mi} ho \ochord{F#m}\censor{saje} a co mi \ochord*{Dsus2}\censor{podrž í}. \\
	Malinký \censor{flíček} na \censor{kalhotách} vydržím, \\
	\censor{kalhot} dost maje, tyto poté \censor{vydražím}.
\end{verse}
\begin{refren}
   
\end{refren}
\end{song}
\pagebreak
