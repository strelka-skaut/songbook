\begin{song}{}
\mysong{Těšínská}{Jaromír Nohavica 1994}{1/0}
\begin{verse}
   Kd^{Ami}ybych se narodil př^{Dmi}ed~sto lety v^{F}~t^{E}omhle m^*{Ami}ěstě^{Dmi}~,~~^{F}  ^{E}  ^{Ami} \\
   u Larischů na zahradě ^{Dmi}trhal bych kv^*{F}ět ^{E}y~své^{Ami}~nev^*{Dmi}ěstě^{F}~,^{E}  ^{Ami} \\
   m^{C}oje nevěsta by b^{Dmi}yla~dcera ševcova \\
   z d^{F}omu Kamiňskich ^{C}odněkud ze Lvova, \\
   kochal bym ja i p^{Dmi}ieščil, ch^*{F}yb^{E}~a lat d^*{Ami}wieš č^{Dmi}ie.~^{F}  ^{E}  ^{Ami} 
\end{verse}
\begin{verse}
   Bydleli bychom na Sachsenbergu v domě u žida Kohna, \\
   nejhezčí ze všech těšínských šperků byla by ona, \\
   mluvila by polsky a trochu česky, \\
   pár slov německy, a smála by se hezky, \\
   jednou za sto let zázrak se koná, zázrak se koná. 
\end{verse}
\begin{verse}
   Kdybych se narodil před sto lety, byl bych vazačem knih \\
   u Prohazků dělal bych od 5 do 5 a 7 zlatek za to bral bych, \\
   měl bych krásnou ženu a tři děti, \\
   zdraví bych měl a bylo by mi kolem třiceti, \\
   celý dlouhý život před sebou, celé krásné dvacáté století. 
\end{verse}
\begin{verse}
   Kdybych se narodil před sto lety v jinačí době, \\
   u Larischů na zahradě trhal bych květy, má lásko, tobě, \\
   tramvaj by jezdila přes řeku nahoru, \\
   slunce by zvedalo hraniční závoru \\
   a z oken voněl by sváteční oběd. 
\end{verse}
\begin{verse}
   Večer by zněla od Mojzese melodie dávnověká, \\
   bylo by léto tisíc devět set deset, za domem by tekla řeka, \\
   vidím to jako dnes: šťastného sebe, \\
   ženu a děti a těšínské nebe, \\
   ještě že člověk nikdy neví, co ho čeká, 
\end{verse}
\begin{deco}
       ^{Ami}na~na n^{Dmi}a~..^{F}.~^{E}  ^{Ami}    ^{Dmi}    ^{F}  ^{E}  ^{Ami} 
\end{deco}
\end{song}
\pagebreak
