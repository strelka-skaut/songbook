\begin{song}{}
\mysong{Sáro!}{Traband 2005}{1/0}
\begin{refren}
   ^{Ami}Sáro, ^{Emi}Sáro, ^{F}v~noci se mi ^{C}zdálo, \\
   že ^{F}tři andělé ^{C}Boží k nám ^*{F}př iš^{C}li na^{G}~oběd \\
   ^{Ami}Sáro, ^{Emi}Sáro, jak ^{F}moc a nebo ^{C}málo \\
   mi ^{F}chybí, abych ^{C}tvojí duši ^{F}mohl ^*{C}ro z^{G}umět? 
\end{refren}
\begin{verse}
   Sbor kajícných mnichů jde krajinou v tichu \\
   a pro všechnu lidskou pýchu \\
   má jen přezíravý smích \\
   A z prohraných válek se vojska domů vrací \\
   Však zbraně stále burácí \\
   a bitva zuří v nich 
\end{verse}
\begin{refren}
   Sáro, Sáro\ldots
\end{refren}
\begin{verse}
   Vévoda v zámku čeká na balkóně \\
   až přivedou mu koně, a pak mává na pozdrav \\
   A srdcová dáma má v každé ruce růže \\
   Tak snadno pohřbít může sto urozených hlav 
\end{verse}
\begin{refren}
   Sáro, Sáro\ldots
\end{refren}
\begin{verse}
   Královnin šašek s pusou od povidel \\
   sbírá zbytky jídel a myslí na útěk \\
   V podzemí skrytí, slepí alchymisté \\
   už objevili jistě proti povinnosti lék
\begin{refren}
   Sáro, Sáro\ldots
\end{refren}
\begin{verse}
   Páv pod tvým oknem zpívá, sotva procit' \\
   o tajemstvích noci ve tvých zahradách \\
   A já - potulný kejklíř, co svázali mu ruce \\
   teď hraju o tvé srdce a chci mít tě na dosah
\begin{refren}
   Sáro, Sáro, pomalu a líně \\
   s hlavou na tvém klíně chci se probouzet \\
   Sáro, Sáro, Sáro, Sáro, rosa padá ráno \\
   a v poledne už možná bude jiný svět \\
   Sáro, Sáro, vstávej, milá Sáro! \\
   Andělé k nám přišli na oběd 
\end{refren}
\end{song}
