\begin{song}{}
\mysong{Lou z Lille}{Klíč 1997}{1/0}
\begin{verse}
   Ona jm^{C}enuje se L^{F}ou a p^{C}ochází prý z L^{G}ille, \\
   má č^{C}ertů rohy, kř^{F}ídla andělů, p^{C}ůvab l^{G}esních v^{C}íl, \\
   kdo strávil s ní pár chv^{F}il, jak š^{C}ampaňské by p^{G}il, \\
   má ^{C}úsměv tupců, tr^{F}ubců závratě pr^{C}o~ni, pr^{G}o~Lou z L^{C}ille. 
\begin{refren}
   Jako ^{C}srpky luny boky ^{F}tenké má a ^{C}úsměv velkých ^{G}dam, \\
   tak se ^{C}lehce vz^{F}náší ^{C}nad ze^{G}mí, letí, ^{C}vůbec ^{G}netuší, ^{C}kam, \\
   jako srpky luny boky ^{F}tenké má a ^{C}úsměv velkých ^{G}dam, \\
   tak se ^{C}lehce vz^{F}náší ^{C}nad ze^{G}mí, letí, ^{C}vůbec ^{G}neví, ^{C}kam. 
\end{refren}
\begin{deco}
   ^{C}Dětsky vážný h^{F}las, větrem^{C}~urousaný v^{G}las \\
   a ^{C}oči jako ^{F}okna za plotem č^*{C}er no^{G}černých ^{C}řas, \\
   co j^{D}á~vím, nemá d^{G}ům, ale ^{D}asi ani b^{A}yt, \\
   a př^{D}esto každý kl^{G}uk chce náramně t^{D}am, kde ^{A}ona, b^{D}ýt. 
\end{deco}
\begin{refren}
   Jako srpky luny boky tenké má a úsměv velkých dam, 
\end{refren}
\begin{deco}
   Víno, vejce, sýr, taky čerstvých ryb dost má, \\
   tohle do košíku každý na trhu zadarmo jí dá, \\
   a báby závidí, mají každé ráno zlost, \\
   a my si ji tu pěkně hýčkáme jen tak, pro radost. 
\begin{deco}
   Jako srpky luny boky tenké má a úsměv velkých dam, 
\end{refren}
\begin{deco}
   Ona jmenuje se Lou a pochází prý z Lille, \\
   má čertů rohy, křídla andělů, půvab lesních víl\ldots 
\end{deco}
\end{song}
\pagebreak
