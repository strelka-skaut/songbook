\mysong{Těšínská}{Jaromír Nohavica }{1994}{}
\begin{song}{}
\begin{verse}
   \mchord{Ami}Kdybych se narodil \\ \mchord{Dmi}před sto lety\mchord{F}  \mchord{E}~v tomhle \mchord*{Ami}měs tě,\mchord{Dmi}~~~ \mchord{F}  \mchord{E}  \mchord{Ami}    \\
	\mchord{Ami}u~Larischů na zahradě \\ \mchord{Dmi}trhal bych květy\mchord{F}~~ \mchord{E}své ne\mchord*{Ami}věstě, \mchord{Dmi}~~~ \mchord{F}  \mchord{E}  \mchord{Ami}    \\
   \mchord{C}moje nevěsta by byla \mchord{Dmi}dcera ševcova \\
   z \mchord{F}domu Kamiňskich \mchord{C}odněkud ze Lvova, \\
   kochal bym ja i \mchord{Dmi}pieščil, c\mchord*{F}hy \mchord{E}ba lat d\mchord*{Ami}wie ščie.\mchord{Dmi}~~~ \mchord{F}  \mchord{E}  \mchord{Ami}   
\end{verse}
\begin{verse}
   Bydleli bychom na Sachsenbergu v domě u žida Kohna, \\
   nejhezčí ze všech těšínských šperků byla by ona, \\
   mluvila by polsky a trochu česky, \\
   pár slov německy, a smála by se hezky, \\
   jednou za sto let zázrak se koná, zázrak se koná.
\end{verse}
\begin{verse}
   Kdybych se narodil před sto lety, byl bych vazačem knih \\
   u Prohazků dělal bych od 5 do 5 a 7 zlatek za to bral bych, \\
   měl bych krásnou ženu a tři děti, \\
   zdraví bych měl a bylo by mi kolem třiceti, \\
   celý dlouhý život před sebou, celé krásné dvacáté století.
\end{verse}
\begin{verse}
   Kdybych se narodil před sto lety v jinačí době, \\
   u Larischů na zahradě trhal bych květy, má lásko, tobě, \\
   tramvaj by jezdila přes řeku nahoru, \\
   slunce by zvedalo hraniční závoru \\
   a z oken voněl by sváteční oběd.
\end{verse}
\begin{verse}
   Večer by zněla od Mojzese melodie dávnověká, \\
   bylo by léto tisíc devět set deset, za domem by tekla řeka, \\
   vidím to jako dnes: šťastného sebe, \\
   ženu a děti a těšínské nebe, \\
   ještě že člověk nikdy neví, co ho čeká, \\
   \mchord{Ami}na~na na\ldots \mchord{Dmi}~~~ \mchord{F}  \mchord{E}  \mchord{Ami}~~~ \mchord{Dmi}~~~ \mchord{F}  \mchord{E}  \mchord{Ami}   
\end{verse}
\end{song}
\pagebreak
