\mysong{Kometa}{Jaromír Nohavica 1986}{}
\begin{song}{}
\begin{verse}
   \mchord{Ami}Spatřil jsem kometu, oblohou letěla, \\
   chtěl jsem jí zazpívat, ona mi zmizela, \\
   zm\mchord{Dmi}izela jako laň \mchord{G7}u~lesa v remízku, \\
   v \mchord{C}očích mi zbylo jen p\mchord{E7}ár~žlutých penízků.
\end{verse}
\begin{verse}
   Penízky ukryl jsem do hlíny pod dubem, \\
   až příště přiletí, my už tu nebudem, \\
   my už tu nebudem, ach, pýcho marnivá, \\
   spatřil jsem kometu, chtěl jsem jí zazpívat.
\end{verse}
\begin{refren}
   \mchord{Ami}O~vodě, o trávě, \mchord{Dmi}o~lese, \\
   \mchord{G7}o~smrti, se kterou smířit \mchord{C}nejde se, \\
   \mchord{Ami}o~lásce, o zradě, \mchord{Dmi}o~světě \\
   \mchord{E}a~o všech lidech, co \mchord{E7}kdy žili na téhle \mchord{Ami}planetě.
\end{refren}
\begin{verse}
   Na hvězdném nádraží cinkají vagóny, \\
   pan Kepler rozepsal nebeské zákony, \\
   hledal, až nalezl v hvězdářských triedrech \\
   tajemství, která teď neseme na bedrech.
\end{verse}
\begin{verse}
   Velká a odvěká tajemství přírody, \\
   že jenom z člověka člověk se narodí, \\
   že kořen s větvemi ve strom se spojuje \\
   a krev našich nadějí vesmírem putuje.
\end{verse}
\begin{refren}
   Na na na\ldots
\end{refren}
\begin{verse}
   Spatřil jsem kometu, byla jak reliéf \\
   zpod rukou umělce, který už nežije, \\
   šplhal jsem do nebe, chtěl jsem ji osahat, \\
   marnost mne vysvlékla celého donaha.
\end{verse}
\begin{verse}
   Jak socha Davida z bílého mramoru \\
   stál jsem a hleděl jsem, hleděl jsem nahoru, \\
   až příště přiletí, ach, pýcho marnivá, \\
   já už tu nebudu, ale jiný jí zazpívá.
\end{verse}
\begin{refren}
   O vodě, o trávě, o lese, \\
   o smrti, se kterou smířit nejde se, \\
   o lásce, o zradě, o světě, \\
   bude to písnička o nás a kometě\ldots
\end{refren}
\end{song}
\pagebreak
