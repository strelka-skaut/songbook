\mysong{Darmoděj}{Jaromír Nohavica 1988}{}
\begin{song}{}
\begin{verse}
   \mchord{Ami}~Šel~včera městem \mchord{Emi}muž~a~šel po hlavní \mchord*{Ami}třídě, \mchord{Emi}    \\
   \mchord{Ami}~šel~včera městem \mchord{Emi}muž~a~já ho z okna \mchord*{Ami}viděl, \mchord{Emi}    \\
   \mchord{C}~na flétnu chorál \mchord{G}hrál, znělo to jako \mchord{Ami}zvon \\
   a byl v tom všechen \mchord{Emi}žal,~ten krásný dlouhý \mchord{F}tón, \\
   a já jsem náhle \mchord{Ddim}věděl:~ano, to je \mchord{E7}on, to je \mchord{Ami}on.
\end{verse}
\begin{verse}
   Vyběh' jsem do ulic jen v noční košili, \\
   v odpadcích z popelnic krysy se honily \\
   a v teplých postelích lásky i nelásky \\
   tiše se vrtěly rodinné obrázky, \\
   a já chtěl odpověď na svoje otázky, otázky.
\end{verse}
\begin{refren}
   |: \mchord{Ami}Na,~na na\mchord*{Emi}\ldots~    ~\mchord{C}  \mchord{G}  \mchord{Ami}    \mchord{F}  \mchord{H}  \mchord{E7}   :|
\end{refren}
\begin{verse}
   Dohnal jsem toho muže a chytl za kabát, \\
   měl kabát z hadí kůže, šel z něho divný chlad, \\
   a on se otočil, a oči plné vran, \\
   a jizvy u očí, celý byl pobodán, \\
   a já jsem náhle věděl, kdo je onen pán, onen pán.
\end{verse}

\chordbreak
\compactbreak

\begin{verse}
   \mchord{Ami}~Celý~se strachem \mchord{Emi}chvěl, když jsem tak k němu \mchord*{Ami}došel, \mchord{Emi}    \\
   \mchord{Ami}~a~v~ústech flétnu \mchord{Emi}měl~od Hieronyma \mchord*{Ami}Bosche \mchord{Emi},   \\
   \mchord{C}~stál měsíc nad do\mchord{G}my~jak čírka ve vo\mchord{Ami}dě, \\
   jak moje svědo\mchord{Emi}mí,~když zvrací v zácho\mchord{F}dě, \\
   a já jsem náhle \mchord{Ddim}věděl:~to je Darmo\mchord{E7}děj, můj Darmo\mchord{Ami}děj.
\end{verse}
\begin{refren}
   \mchord{Ami}Můj Darmo\mchord{Emi}děj, vaga\mchord{C}bund osu\mchord{G}dů~a lásek, \\
   \mchord{Ami}~jenž prochá\mchord{F}zí~všemi \mchord{H}sny, ale \mchord{E7}dnům vyhýbá se, \\
   \mchord{Ami}můj Darmo\mchord{Emi}děj,~krásné \mchord{C}zlo, jed má \mchord{G}pod jazykem, \\
   \mchord{Ami}~když prodá\mchord{F}vá po do\mchord{H}mech jehly \mchord{E7}se~slovníkem.
\end{refren}
\begin{verse}
   Šel včera městem muž, podomní obchodník, \\
   šel, ale nejde už, krev skápla na chodník, \\
   já jeho flétnu vzal a zněla jako zvon \\
   a byl v tom všechen žal, ten krásný dlouhý tón, \\
   a já jsem náhle věděl: ano, já jsem on, já jsem on.
\end{verse}
\begin{refren}
   Váš Darmoděj, vagabund\ldots
\end{refren}

  \begin{chords}
    \chordDdim{}
  \end{chords}
\end{song}
\pagebreak
