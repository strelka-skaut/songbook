\mysong{Zítra ráno v pět}{Jaromír Nohavica 1988}{}
\begin{song}{}
\begin{verse}
   Až \mchord{Emi}mě~zítra ráno v pět \mchord{G}ke~zdi postaví, \\
   \mchord{C}ještě si napo\mchord{D7}sled dám \mchord{G}vodku na zdr\mchord{E}aví. \\
   Z očí\mchord{Ami}~pásku strhnu s\mchord{D7}i,~to abych \mchord{G}viděl na ne\mchord{Emi}be, \\
   a \mchord{Ami}pak~vzpomenu \mchord{H7}si~má l\mchord{Emi}ásko~na tebe.  \mchord{Ami}    \mchord{D7}   \mchord{G}  \mchord{Emi}    \\
   a \mchord{Ami}pak~vzpomenu \mchord{H7}si~na te\mchord{Emi}be.
\end{verse}
\begin{verse}
   Až zítra ráno v pět přijde ke mně kněz, \\
   řeknu mu, že se splet', že mně se nechce do nebes, \\
   že žil jsem, jak jsem žil, a stejně tak i dožiju \\
   a co jsem si nadrobil, to si i vypiju, \\
   a co jsem si nadrobil, si i vypiju.
\end{verse}
\begin{verse}
   Až zítra ráno v pět poručík řekne: \enquote{Pal!},  \\
   škoda bude těch let, kdy jsem tě nelíbal, \\
   ještě slunci zamávám, a potom líto přijde mi, \\
   že tě, lásko, nechávám, samotnou tady na zemi, \\
   že tě, lásko, nechávám, na zemi.
\end{verse}
\begin{verse}
   Až zítra ráno v pět prádlo půjdeš prát \\
   a seno obracet, já u zdi budu stát, \\
   tak přilož na oheň a smutek v sobě skryj, \\
   prosím, nezapomeň, nezapomeň a žij, \\
   na mě nezapomeň a žij\ldots 
\end{verse}
\end{song}
\pagebreak
