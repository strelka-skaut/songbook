\mysong{Jacek}{Jaromír Nohavica }{1989}{\fourfour}
\begin{song}{}
\begin{verse}
   \mchord{G}~Na~druhém břehu řeky \mchord{D}Olše žije Jacek, \\
   \mchord{C}~mám k němu stejně blízko \mchord{G}jak on ke mně, \\
   \hchord{G}~máváme na sebe z \hchord{D}říční navigace, \\
   \hchord{C}~dva~spojenci a dvě \hchord{G}spřátelené země, \\
   jak malí kluci hážem z \hchord{D}břehů žabky, \\
   \hchord{C}kdo vyhraje, má z pro\hchord{G}tějšího srandu, \\
   hlavama kroutí česko-\hchord{D}polské babky, \\
   \hchord{C}děláme prostě vlastní \hchord{G}propagandu.
\end{verse}
\begin{verse}
  \mchord{G}~Na mostě přátelství se \mchord{D}tvoří dlouhé fronty \\
   \mchord{C}~všelikých věcí za vše\mchord{G}likou cenu, \\
   já mám však na to velmi úzké horizonty \\
   a Jacek velmi nenáročnou ženu, \\
   týden co týden z břehů navigace \\
   na sebe řveme:"Chlapče, hlavu vzhůru!", \\
   jak je to krásné, moci vykašlat se \\
   na celní předpisy a na cenzuru, na na\ldots
\end{verse}
\pagebreak
\begin{verse}
  \mchord{G}~Z Piastovské věže na nás \mchord{D}mává kníže Měšek \\
   \mchord{C}~a~směje se, až třepe \mchord{G}se mu brada, \\
   \hchord{G}~ve zprávách večer běží \hchord{D}horký dnešek, \\
   \hchord{C}~aspoň se máme s Jackem \hchord{G}o~co hádat, \\
   on tvrdí svoje, já zas tvrdím svoje \\
   a domluvit se někdy bývá marno, \\
   tak spolu vedem pohraniční boje \\
   a v praxi demonstrujem Solidarnosc, na na\ldots
\end{verse}
\begin{verse}
  \mchord{G}~Na druhém břehu řeky \mchord{D}Olše žije Jacek, \\
   \mchord{C}~mám k němu stejně blízko \mchord{G}jak on ke mně, \\
   máváme na sebe z říční navigace, \\
   dva spojenci a dvě spřátelené země \\
   |: a voda plyne, plyne, plyne dlouhé věky, \\
      řeka se kroutí jako modrá šňůrka \\
      a my dva hážem kachnám vprostřed řeky \\
      krajíčky chleba o dvou stejných kůrkách. :| \\
   Na na na\ldots 
\end{verse}
\end{song}
\pagebreak
