\begin{song}{}
\mysong{Dokud se zpívá}{Jaromír Nohavica 2006}{1/0}
\begin{verse}
   ^{C}Z~Těšína ^{Emi}vyjíždí ^{Dm7}vlaky co ^*{F}čt vrthod^*{C}in u^{Emi},  ^{Dm7} ^{G} \\
   ^{C}včera jsem ^{Emi}nespal a ^{Dm7}ani~dnes ^*{F}ne spoč^*{C}in u^{Emi},  ^{Dm7} ^{G} \\
   ^{F}svatý Me^{G}dard, můj pa^{C}tron, ťuká si na če^{G}lo, \\
   ale ^{F}dokud se ^{G}zpívá, ^{F}ještě se ^*{G}ne umř^*{C}el o^{Emi}. ^{Dm7} ^{G} 
\end{verse}
\begin{verse}
   Ve stánku koupím si housku a slané tyčky, \\
   srdce mám pro lásku a hlavu pro písničky, \\
   ze školy dobře vím, co by se dělat mělo, \\
   ale dokud se zpívá, ještě se neumřelo. 
\end{verse}
\begin{verse}
   Do alba jízdenek lepím si další jednu, \\
   vyjel jsem před chvílí, konec je v nedohlednu, \\
   za oknem míhá se život jak leporelo, \\
   ale dokud se zpívá, ještě se neumřelo. 
\end{verse}
\begin{verse}
   Stokrát jsem prohloupil a stokrát platil draze, \\
   houpe to, houpe to na housenkové dráze, \\
   i kdyby supi se slítali na mé tělo, \\
   tak dokud se zpívá, ještě se neumřelo. 
\end{verse}
\begin{verse}
   Z Těšína vyjíždí vlaky až na kraj světa, \\
   zvedl jsem telefon a ptám se: "Lidi, jste tam?" \\
   ^{F}A~z veliké ^{G}dálky do ^{C}uší mi zazně^{G}lo, \\
   |: že ^{F}dokud se ^{G}zpívá, ^{F}ještě se ^*{G}ne umř^*{C}el o^{Emi}. ^{Dm7} ^{G} :|
\end{verse}
\begin{solo}F        G      F        G     C   Emi   Dm7   G\end{solo}
\end{song}
\pagebreak
