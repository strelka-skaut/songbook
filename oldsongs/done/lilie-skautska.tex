\begin{song}{}
\mysong{Lilie Skautská}{Taxmeni 1991}{1/0}
\begin{verse}
   Na léta ^{D}klukovský si ^{A}často vzpomí^{G}nám, \\
   na první ^{D}skautskej slib, na jeho krásný ^*{A}sl ova, \\
   a v duchu s ^{D}bráchou zas pod ^{A}stanem usí^{G}nám, \\
   skončil den v ^{D}táboře, a do ^{A}tmy houká ^{D}sova. 
\end{verse}
\begin{verse}
   Novej rok přines' mráz, Únor, a v srdcích led, \\
   zlej vítr zafoukal ze stepi od východu, \\
   sebral nám naději, ne, nebude to hned, \\
   stát za svým názorem, a nejít do průvodu. 
\end{verse}
\begin{refren}
   Lilie skautská, ohněm kalená, máme ji v sobě, v nás nikdy nezvadne, \\
   dej Bůh, ať hvězda krví zbarvená už jednou provždy zapadne. 
\end{refren}
\begin{verse}
   Poslední prázdniny bratr řek' akorát  \\
   skautskýmu slibu, že zůstanem věrní \\
   a tak nám na podzim přišili na kabát  \\
   vojenský výložky, jak saze černý. 
\end{verse}
\begin{verse}
   Když jsme pak po vojně svobodní chtěli být, \\
   zbylo nám jediný, na západ přejít čáru, \\
   novýho bobříka odvahy ulovit,  \\
   usínat pod stanem a hrávat na kytaru. 
\end{verse}
\begin{refren}
   Lilie skautská\ldots 
\end{refren}
\begin{verse}
   Vidím tu krutou noc, vidím vše jako dnes, \\
   plazím se po břiše, kolem nás ostnatej drát, \\
   zableskly výstřely a po krku skočil pes, \\
   tys bratře odešel do věčnejch lovišť spát. 
\end{verse}
\begin{verse}
   To už mi nebylo, bráško můj, do tance,  \\
   ani jsem nemoh' ti naposled zamávat, \\
   na rukách pouta a k tomu kopance,  \\
   před sebou uran na osm let dolovat. 
\end{verse}
\begin{refren}
   Lilie skautská\ldots 
\end{refren}
\begin{verse}
      Lilie skautská, v ohni kalená, máme ji v srdcích, tam nikdy nezvadne, \\
      dej Bůh, ať hvězda krví zbrocená, do bláta už navždy zapadne. \\
\end{verse}
\begin{refren}
   Lilie skautská\ldots 
\end{refren}
\end{song}
\pagebreak
